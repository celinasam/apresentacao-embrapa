 \section{Introdução}

  \begin{frame}{Motivação}
     \begin{itemize}
      \item Este trabalho é uma contribuição para \emph{software} livre em sistemas distribuídos.
      \item Armazenamento de arquivos é um componente essencial na computação de alto desempenho. 
      \item Códigos Corretores de Erro (\emph{Erasure codes}) introduzem redundância e tem sido utilizados em sistemas para alcançar confiabilidade e redução do custo de armazenamento.    
     \end{itemize}
  \end{frame}

  \begin{frame}{Motivação}
     \begin{itemize}
      \item Alguns sistemas que utilizam códigos corretores de erros:
	\begin{itemize}
%           \item \emph{Digital Fountain} (\emph{multicasting} multimídia confiável)\cite{Byers:1998}
            \item \emph{NASA's Deep Space Network} no envio e na recepção de sinais e dados de telemetria
(\emph{downlinks}) vindos de veículos espaciais (\emph{very distant spacecrafts}) e para enviar telecomandos (\emph{uplinks}) para
veículos espaciais \cite{Abrantes:2010, Almeida:2007, STO:2010, TDD:2010};
           \item \emph{Delay and Disruption Tolerant Networks}, redes de sensores e redes~\emph{peer-to-peer} \cite{Bhagwan:2004, Alencar:2004, Haeberlen:2005, Houri:2009,RTAD:2007, Rodrigues:2005, Wilcox-O'Hearn:2008};
           \item armazenamento de grande volume de dados \cite{Anderson:1998,Kubiatowicz:2000,  Saito:2004, Schmuck:2002, Storer:2008,
Storer:2009, Xia:2006}, como também o sistema de arquivos distribuído do Hadoop (HDFS)~\cite{HDFS-503:2010}.
%           \item sistema de arquivos distribuído do Hadoop (HDFS)~\cite{HDFS-503:2010}
	\end{itemize}
     \end{itemize}
  \end{frame}

  \begin{frame}{Motivação}
     \begin{itemize}
      \item O HDFS, por padrão, implementa alta disponibilidade dos dados via replicação simples dos blocos de dados. Esta abordagem acarreta um alto custo de armazenamento para garantir que os dados estarão sempre disponíveis.
      \item Esforços iniciais nessa linha foram feitos utilizando técnicas de \emph{Redundant Array of Independent Drives} (RAID)~\cite{HDFS-503:2010,Patterson:1988} e mais recentemente do algoritmo Reed-Solomon (RS)~\cite{MR-1969:2010}.
     \end{itemize}
  \end{frame}

  \begin{frame}{Motivação}
     \begin{itemize}
        \item RAID
        \item Reed-Solomon
     \end{itemize}
  \end{frame}

  \begin{frame}{Objetivos deste trabalho}

  \begin{itemize}
     \item avaliação de desempenho, ganhos, e custos de diferentes
  estratégias de códigos corretores de erro;
     \item implementação de otimizações ou extensões para o código que
  atualmente implementa Reed-Solomon, tentando melhorar,
  principalmente, a parte de distribuição de blocos;
     \item implementação de novos algoritmos (e.g., Tornado codes) e
  extensão da interface atual para aceitá-los;
     \item integração do código atual com o HDFS.
     \end{itemize}
  \end{frame}
